For simplicity, the algorithms considered in this paper will be presented in a
tree-independent context, as in \citet{curtin2013tree}, but the only type of
tree we will consider is the cover tree \citep{langford2006}, and the only type
of traversal we will consider is the cover tree pruning dual-tree traversal,
which we will describe below.

As we will be making heavy use of trees, we must establish notation \citep[taken
from][]{curtin2013tree}.  The notation we will be using is defined in Table
\ref{tab:notation}.

\begin{table}
{\small
\begin{center}
\begin{tabular}{|c|l|}
\hline
{\bf Symbol} & {\bf Description} \\ \hline
$\mathscr{N}$ & A tree node \\ \hline
$\mathscr{C}_i$ & Set of child nodes of $\mathscr{N}_i$ \\ \hline
$\mathscr{P}_i$ & Set of points held in $\mathscr{N}_i$ \\ \hline
$\mathscr{D}_i^n$ & Set of descendant nodes of $\mathscr{N}_i$ \\ \hline
$\mathscr{D}_i^p$ & Set of points contained in $\mathscr{N}_i$ and
$\mathscr{D}_i^n$ \\ \hline
$\mu_i$ & Center of $\mathscr{N}_i$ (for cover trees, $\mu_i = p_i$) \\ \hline
$\lambda_i$ & Furthest descendant distance \\ \hline
\end{tabular}
\end{center}
}
\caption{Notation for trees.  See \cite{curtin2013tree} for details.}
\label{tab:notation}
\end{table}

The cover tree is a leveled hierarchical data structure originally proposed for
the task of nearest neighbor search.  The original description of the tree
describes both an {\it implicit} and {\it explicit} representation, where the
explicit representation is the actual structure that would be implemented in
practice.  Thus, we will consider only the explicit representation in this
paper.

Each node $\mathscr{N}_i$ in the cover tree is indexed by an integer scale $s_i$
and holds one point $p_i$ which is also the center of the node ($p_i = \mu_i$).
The cover tree node may have a number of child nodes; if the cover tree node
does have children, then one child will be the self-child, which holds the same
point $p_i$ with a smaller scale $s_i$.  Each child of $\mathscr{N}_i$ will have
scale less than $s_i$, and every descendant point of a node $\mathscr{N}_i$
(that is, the set $\mathscr{D}_i^p$) has a maximum distance of $2^{s_i + 1}$
from the point $p_i$.

Assuming a cover tree $\mathscr{T}$ is built on a dataset $S_r$, let $C_{s_i}$
denote the set of points held in all nodes in $\mathscr{T}$ with scale $s_i$.
Then, the cover tree satisfies the following invariants for all scales $s_i$:

\begin{itemize}

\item {\em (Nesting)}.  $C_{s_i} \subset C_{s_i - 1}$.  This means that when a
point $p_i \in S_r$ is held in a node at some scale $s_i$, then each smaller
scale will also have a node containing $p_i$.

\item {\em (Covering tree)}.  For every point $p_i \in C_{s_i - 1}$, there
exists a point $p_j \in C_{s_i}$ such that $d(p_i, p_j) < 2^{s_i}$, and the node
holding $p_j$ with scale $s_i$ is the parent node of the node holding $p_i$ with
scale $s_i - 1$.

\item {\em (Separation)}.  For all distinct points $p_i, p_j \in C_{s_i}$,
$d(p_i, p_j) > 2^{s_i}$.

\end{itemize}

A batch construction algorithm is given by \citet{langford2006}, called
\texttt{Construct}; but the details of construction are not relevant to our
work, so we only mention this as a reference.  More relevant to our work is the
number of useful and unique theoretical properties the cover tree possesses,
based on the expansion constant \citep{karger2002finding}; we restate the
definition below.

\begin{defn}
\label{def:int_dim}
Let $B_S(p, \Delta)$ be the set of points in $S$ within a closed ball of radius
$\Delta$ around some $p \in S$ with respect to a metric $d$:
%
$B_S(p, \Delta) = \{ r \in S \colon d(p, r) \leq \Delta \}$.
%
Then, the {\bf expansion constant} of $S$ with respect to the metric $d$ is the
smallest $c \ge 2$ such that

\begin{equation}
| B_S(p, 2 \Delta) | \le c | B_S(p, \Delta) |\ \forall\ p \in S,\
\forall\ \Delta > 0.
\end{equation}

\end{defn}

The expansion constant is used heavily in the cover tree literature.  It is
related to a notion of instrinic dimensionality, and previous work has shown
that there are many scenarios where $c$ is independent of the number of points
in the dataset.

The expansion constant can be used to show a few useful bounds on various
properties of the cover tree; we restate these results below, given some cover
tree built on a dataset $S$ with expansion constant $c$ and $|S| = N$:

\begin{itemize}
  \item {\bf Width bound:} no cover tree node has more than $c^4$ children
(Lemma 4.1, \cite{langford2006}).

  \item {\bf Depth bound:} the maximum depth of any node is $O(c^2 \log N)$
(Lemma 4.3, \cite{langford2006}).

  \item {\bf Space bound:} a cover tree has $O(N)$ nodes (Theorem 1,
\cite{langford2006}).
\end{itemize}

%\begin{lemma}
%\label{lem:width}
%(Lemma 4.1, \cite{langford2006}) The number of children of any cover tree node $\mathscr{N}_i$ is bounded by
%$c^4$, where $c$ is the expansion constant of the dataset the cover tree is
%built on, as defined in Definition \ref{def:int_dim}.
%\end{lemma}

%\begin{lemma}
%\label{lem:depth}
%(Lemma 4.3, \cite{langford2006}) The maximum depth of any point $p_r$ in a cover
%tree $\mathscr{T}_r$ is $O(c^2 \log N)$, where $N$ is the number of points in
%the dataset that $\mathscr{T}_r$ is built on.
%\end{lemma}

Lastly, we introduce a convenience lemma of our own which is a generalization of
the packing arguments used by \citet{langford2006}.  This is a more flexible
version of their argument.

\begin{lemma}
Consider a dataset $S$ with expansion constant $c$ and a subset $C \subseteq S$
such that every point in $C$ is separated by $\delta$.  Then, given any query
point $p \not\in S$ and some radius $\rho \delta$,

\begin{equation}
| B_S(p, \rho \delta) \cap C | \le c^{2 + \log_2 \rho}.
\end{equation}
\label{lem:packing}
\end{lemma}

\begin{proof}
This is based on the packing argument from Lemma 4.1 in \cite{langford2006}.
Observe that $B_S(p, \rho \delta) \subseteq B_S(p_i, 2 \rho \delta)$ for any
$p_i \in S$, and that $| B_S(p, 2 \rho \delta) | = c^{2 + \log_2 \rho} |
B_S(p, \delta / 2) |$.  Because each point in $C$ is separated by $\delta$, the
number of points in $B_S(p, \rho \delta) \cap C$ is
bounded by the number of disjoint balls of radius $\delta / 2$ that can be
packed into $B_S(p, \rho \delta)$.  In the worst case, this packing is
perfect, and

\begin{equation}
|B_S(p, \rho \delta)| \le \frac{|B_S(p_i, 2 \rho \delta)|}{|B_S(p_i, \delta
/ 2)|} \le c^{2 + \log_2 \rho}.
\end{equation}
\end{proof}
